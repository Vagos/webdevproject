\documentclass[acmtog, nonacm, language=english, language=greek]{acmart}

\usepackage{tikz}
\usetikzlibrary{shapes.geometric, arrows}

\tikzstyle{startstop} = [rectangle, rounded corners, minimum width=3cm, minimum height=1cm,text centered, draw=black, fill=red!30]
\tikzstyle{io} = [trapezium, trapezium left angle=70, trapezium right angle=110, minimum width=3cm, minimum height=1cm, text centered, draw=black, fill=blue!30]
\tikzstyle{process} = [rectangle, minimum width=3cm, minimum height=1cm, text centered, draw=black, fill=orange!30]
\tikzstyle{decision} = [diamond, minimum width=3cm, minimum height=1cm, text centered, draw=black, fill=green!30]
\tikzstyle{arrow} = [thick,->,>=stealth]


\newcommand{\en}[1]{\textlatin{#1}}

%%
%% \BibTeX command to typeset BibTeX logo in the docs
\AtBeginDocument{%
  \providecommand\BibTeX{{%
    Bib\TeX}}}

%% Rights management information.  This information is sent to you
%% when you complete the rights form.  These commands have SAMPLE
%% values in them; it is your responsibility as an author to replace
%% the commands and values with those provided to you when you
%% complete the rights form.
\setcopyright{acmcopyright}
\copyrightyear{2022}
\acmYear{2022}
\acmDOI{}



%% If you are preparing content for an event
%% sponsored by ACM SIGGRAPH, you must use the "author year" style of
%% citations and references.
\citestyle{acmauthoryear}



%%
%% end of the preamble, start of the body of the document source.
\begin{document}

%%
%% The "title" command has an optional parameter,
%% allowing the author to define a "short title" to be used in page headers.
\title{Σχεδιασμός και Υλοποίηση Ιστοσελίδας Υποστήριξης Κοινότητας}

%%
%% The "author" command and its associated commands are used to define
%% the authors and their affiliations.
%% Of note is the shared affiliation of the first two authors, and the
%% "authornote" and "authornotemark" commands
%% used to denote shared contribution to the research.
\author{Ευάγγελοσ Λάμπρου}
\email{e.lamprou@upnet.gr}
\orcid{}
\affiliation{%
  \institution{Πανεπιστήμιο Πατρών}
  \streetaddress{}
  \city{}
  \state{}
  \country{}
  \postcode{}
}

\author{Απόστολοσ Παπαδημητρίου}
\affiliation{%
  \institution{Πανεπιστήμιο Πατρών}
  \streetaddress{}
  \city{}
  \country{}}
\email{a.papadimitriou@upnet.gr}

%%
%% The abstract is a short summary of the work to be presented in the
%% article.
\begin{abstract}
    Στα πλαίσια αυτής της εργασίας γίνεται ο σχεδιασμός και η υλοποίηση μίας 
    ιστοσελίδας Υποστήριξης Κοινότητας. Γίνεται ανάλυση των απαραίτητων 
    λειτουργιών για την εφαρμογή και περιγράφεται το σύνολο των τεχνολογιών 
    που χρησιμοποιήθηκαν με εστίαση σε αξιοσημείωτα μέρη της υλοποίησης.
\end{abstract}

\maketitle

\section{Εισαγωγή}

\section{Ενοιολογικόσ Σχεδιασμόσ}

\subsection{Λειτουργικές Απαιτήσεις}
\subsection{Βάση Δεδομένων}
\subsection{Διάγραμμα Ροής Χρήστη}

\begin{tikzpicture}[node distance=3.5cm, scale=0.6, every/.style={transform shape}]

    \node (start) [startstop] {Αρχική Σελίδα};

    \node (loggedin) [decision, below of=start] {\footnotesize Συνδεδεμένος?};
    \node (activitiespage) [process, left of=loggedin] {Δραστηριοτήτες};
    \node (postspage) [process, below of=activitiespage] {Αναρτήσεις};
    \node (postpage) [process, below of=postspage] {Ανάρτηση};
    \node (profile) [process, below of=postpage] {Προφίλ Χρήστη};
    \node (loginpage) [process, below of=loggedin] {Σύνδεση};
    \node (registered) [decision, below of=loginpage] {\footnotesize Εγγεγραμένος?};
    \node (loginsucess) [decision, below of=registered] {\footnotesize Επιτυχια?};
    \node (registerpage) [process, right of=registered] {Εγγαρφή};
    \node (personalprofile) [process, right of=loggedin] {Προσωπικό προφίλ};

    \draw [arrow] (start) -- (loggedin);
    \draw [arrow] (start) -| (activitiespage);
    \draw [arrow] (loggedin) -- node[right] {όχι} (loginpage);
    \draw [arrow] (loggedin) -- node[above] {ναι} (personalprofile);
    \draw [arrow] (loginpage) -- (registered);
    \draw [arrow] (registered) -- node[above] {όχι} (registerpage);
    \draw [arrow] (registered) -- node[right] {ναι} (loginsucess);
    \draw [arrow] (loginsucess) -- node[right] {ναι} (personalprofile);
    \draw [arrow] (registerpage) |- (loginpage);
    \draw [arrow] (activitiespage) -- (postspage);
    \draw [arrow] (postspage) -- (postpage);
    \draw [arrow] (postpage) -- (profile);

\end{tikzpicture}

\section{Υλοποίηση}

\subsection{Τεχνολογίες}
\subsection{Λειτουργικότητα Εφαρμογής}

\end{document}
\endinput
%%
%% End of file `sample-acmtog.tex'.
